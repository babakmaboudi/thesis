\chapter{Manifolds and Differential Forms}

In preparation for later chapters, we recall basic facts regarding differential geometry and symplectic geometry. Although the applications of MOR presented in this thesis are mostly described on a linear vector space, the theory of smooth manifold can provide a great insight into the geometric MOR. The main aim of this chapter is to introduce the concepts of ``symmetry'' and ``structure'' which play an important role in the Lagrangian and Hamiltonian mechanics. Exploiting these symmetries and structures in MOR can provide robustness and long-time stability in the reduced system.

\section{Smooth Manifolds}
Let $\mathcal M$ to be a topological Hausdorff \cite{friedman1970foundations} set. In the neighborhood $U$ of $p\in \mathcal M$, if there is a bijective map $x:U\to V\subset \mathbb R^{m}$, for some positive integer $m$, then $(x,U)$ is called a \emph{coordinate chart}. $\mathcal M$ is called a \emph{smooth manifold} if there exists a set of coordinate charts $\{(x_{\alpha},U_{\alpha})\}_{\alpha\in I}$ such that $\{U_{\alpha}\}_{\alpha\in I}$ covers $\mathcal M$ and that the mapping $x_{\alpha_2}\circ (x_{\alpha_1})^{-1}:\mathbb R^{m} \to \mathbb R^{m}$ is a $C^{\infty}$ mapping, for any $\alpha_1,\alpha_2\in I$. The integer $m$ is called the \emph{dimension} of $\mathcal M$ if $x_{\alpha}(U_{\alpha}) \subset \mathbb R^{m}$ for all $\alpha \in I$. Throughout this thesis, we assume that there exists a coordinate chart $(x,U)$ that is global, i.e., $U$ covers $\mathcal M$.

Tangent vectors to a manifold allow us to describe the local dynamics of a particle moving on a smooth manifold. There are multiple ways to define tangent vectors. The most intuitive way however, uses curves defined on a smooth manifold.

A $C^{\infty}$ mapping $\gamma:\mathbb R \to \mathcal M$ is called a curve on $\mathcal M$ passing through $p\in \mathcal M$ if $\gamma(t) = p$, for some $t\in \mathbb R$. Without loss of generality, we may assume that $\gamma$ passes through $p$ at $t=0$. We define an equivalence relation between smooth curves that pass through $p$ as follows: $\gamma_1$ and $\gamma_2$, two smooth curves on $\mathcal M$, are equivalent if
\begin{equation} \label{eq:2.1}
	\frac{d}{dt}(x\circ \gamma_1)|_{t=0} = \frac{d}{dt}(x\circ \gamma_2)|_{t=0},
\end{equation}
for a coordinate chart $x$. It is checked easily that this definition is chart independent, i.e., the equivalent classes do not depend on the choice of the coordinate chart.
\begin{definition}
A tangent vector $v$ at a point $p\in \mathcal M$ is an equivalence class of curves on $\mathcal M$ that pass through $p$. The set of all tangent vectors at $p$ is denoted as $T_p\mathcal M$ and is called the tangent space of $\mathcal M$ at point $p$.
\end{definition}
It is well known that the $T_p\mathcal M$ forms an $m$-dimensional linear vector space \cite{robbin2011introduction}. It is shown in \cite{robbin2011introduction} that the vector space $T_p\mathcal M$ is isomorphic to the vector space of linear differential operators, directional derivatives, acting on the differentiable real-valued functions defined on an open neighborhood of $p$. Therefore, a basis for $T_p\mathcal M$ is often denoted as $\{ \partial /\partial x_i \}_{i=1}^{m}$ and a vector $v\in T_p\mathcal M$ can be written as 
\begin{equation}
	\sum_{i=1}^m v_i \frac{\partial}{\partial x_i},
\end{equation}
where $v_i\in \mathbb R$ for $i = 1,\dots,m$. Note that the symbol for partial derivative is rather symbolic since the direction $x_i$ is not defined explicitly on $T_p\mathcal M$. To compute a tangent vector for a given chart $x$ and a given smooth function $f:\mathcal M \to \mathbb R$ one reads
\begin{equation}
	\sum_{i=1}^m v_i \frac{\partial}{\partial x_i} (f) := \sum_{i=1}^m v_i \frac{\partial}{\partial x_i} (f \circ x^{-1}),
\end{equation}
where the partial derivatives appear on the right hand side are the conventional partial derivative operators on a Euclidean space.

It is often desirable to consider the dual space to $T_p\mathcal M$, denoted by $T_p^*\mathcal M$ and referred to as the \emph{cotangent space}. The natural isomorphism between $T_p\mathcal M$ and $T_p^*\mathcal M$ indicates that the cotangent space is also an $m$-dimensional linear vector space. Given $\{ \partial /\partial x_i \}$, a basis for $T_p\mathcal M$, a dual basis for $T_p^*\mathcal M$ is the set of basis vectors $\{ dx_i \}_{i=1}^{m}$ that satisfy the following
\begin{equation}
	dx_i(\frac{\partial}{\partial x_j}) = \delta_{i,j}, \quad i,j=1,\dots,m,
\end{equation}
where $\delta_{i,j}$ is the Kronecker's delta function.

To be able to define a vector field on a manifold, we need to assign a tangent vector to every point of a manifold. Such an object belongs to a structure that, informally, is obtained by glueing the tangent space $T_p\mathcal M$ to every point $p\in \mathcal M$. This structure is referred to as the \emph{tangent bundle}, denoted as $T \mathcal M$, and is defined as $T \mathcal M := \{ (p,v) | p\in \mathcal M, \ v \in T_p \mathcal M\}$.
\begin{theorem}
\cite{robbin2011introduction} The tangent bundle $T\mathcal M$ is a smooth manifold.
\end{theorem}

\begin{proof}
We define the projection operator $\pi:T\mathcal M \to M$ as
\begin{equation} \label{eq:2.2}
	\pi : (p,v) \to p.
\end{equation}
It is checked easily that $\pi^{-1}(\{p\})$ is the $m$-dimensional linear vector space $T_{p} \mathcal M$. Now assume that $(x,U)$ is a coordinate chart for $\mathcal M$, such that $p\in U$. We construct a coordinate chart $(\bar x , \pi^{-1}(U) )$ for $T\mathcal M$ as follows 
\begin{equation} \label{eq:2.3}
	\begin{aligned}
		&\bar x : \pi^{-1}(U) \subset T \mathcal M \to \mathbb R^m \times \mathbb R^m, \\
		&\bar x : (p,v) \to (x(p),v_1,\dots,v_m),
	\end{aligned}
\end{equation}
where $v_1,\dots,v_m\in \mathbb R$ are components of $v$ in $\mathbb R^m$. It can be checked easily that for two intersecting coordinate charts $(x_\alpha,U_\alpha)$ and $(x_\beta, U_\beta)$, the transition map $\bar x_{\alpha} \circ (\bar x_\beta)^{-1}$ is a $C^{\infty}$ map. Thus, $T \mathcal M$ is a smooth manifold, and the mapping (\ref{eq:2.3}) suggests, it is $2m$-dimensional.
\end{proof}

In similar fashion, we may obtain a smooth manifold by gluing the cotangent space $T^*_p \mathcal M$ to the manifold $\mathcal M$ to obtain the \emph{cotangent bundle}, denoted as $T^* \mathcal M$. 

\begin{definition}
continuous injective mappings $X:\mathcal M \to T \mathcal M$ and $X^*:\mathcal M \to T^* \mathcal M$ is called a \emph{vector field} and a \emph{covector field} of $\mathcal M$, respectively.
\end{definition}


In the study of MOR, transformation between vector spaces emerges naturally. In a more general setting, it is beneficial to study transformations between smooth manifolds. Later in this chapter, we discuss how some manifold structures can be preserved over such transformation which lays a basis for geometric MOR.

Let $\mathcal M$ and $N$ be an $m$ dimensional and an $n$ dimensional smooth manifolds, respectively. Let $f:\mathcal M\to N$ be a smooth mapping. Then the \emph{differential map} of $f$ at a point $p\in \mathcal M$, denoted as $T_{p}f$,is a map between the tangent spaces $T_p \mathcal M$ and $T_{f(p)} N$ defined as
\begin{equation} \label{eq:2.4}
	T_p f(v) = \frac{d}{dt}(f\circ \gamma(t))|_{t=0},
\end{equation}
for some tangent vector $v\in T_{p} \mathcal M$ and some curve $\gamma$ in the equivalence class of $v$. It can be shown that $T_p f$ only depends of $v$ and not the choice of the curve $\gamma$ \cite{robbin2011introduction}.
