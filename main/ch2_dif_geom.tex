\chapter{Manifolds and Differential Forms} \label{chapter:2}

In preparation for later chapters, we recall basic facts regarding differential geometry and symplectic geometry. Although the applications of MOR presented in this thesis are mostly described on a linear vector space, the theory of smooth manifold can provide a great insight into the geometric MOR. The main aim of this chapter is to introduce the concepts of ``symmetry'' and ``structure'' which play an important role in the Lagrangian and Hamiltonian mechanics. Exploiting these symmetries and structures in MOR can provide robustness and long-time stability in the reduced system.

\section{Smooth Manifolds} \label{section:2.1}
Let $\mathcal M$ to be a topological Hausdorff \cite{friedman1970foundations} set. In the neighborhood $U$ of $p\in \mathcal M$, if there is a bijective map $x:U\to V\subset \mathbb R^{m}$, for some positive integer $m$, then $(x,U)$ is called a \emph{coordinate chart}. $\mathcal M$ is called a \emph{smooth manifold} if there exists a set of coordinate charts $\{(x_{\alpha},U_{\alpha})\}_{\alpha\in I}$ such that $\{U_{\alpha}\}_{\alpha\in I}$ covers $\mathcal M$ and that the mapping $x_{\alpha_2}\circ (x_{\alpha_1})^{-1}:\mathbb R^{m} \to \mathbb R^{m}$ is a $C^{\infty}$ mapping, for any $\alpha_1,\alpha_2\in I$. The integer $m$ is called the \emph{dimension} of $\mathcal M$ if $x_{\alpha}(U_{\alpha}) \subset \mathbb R^{m}$ for all $\alpha \in I$. Throughout this thesis, we assume that there exists a coordinate chart $(x,U)$ that is global, i.e., $U$ covers $\mathcal M$.

Tangent vectors to a manifold allow us to describe the local dynamics of a particle moving on a smooth manifold. There are multiple ways to define tangent vectors. The most intuitive way however, uses curves defined on a smooth manifold.

A $C^{\infty}$ mapping $\gamma:\mathbb R \to \mathcal M$ is called a curve on $\mathcal M$ passing through $p\in \mathcal M$ if $\gamma(t) = p$, for some $t\in \mathbb R$. Without loss of generality, we may assume that $\gamma$ passes through $p$ at $t=0$. We define an equivalence relation between smooth curves that pass through $p$ as follows: $\gamma_1$ and $\gamma_2$, two smooth curves on $\mathcal M$, are equivalent if
\begin{equation} \label{eq:2.1}
	\frac{d}{dt}(x\circ \gamma_1)|_{t=0} = \frac{d}{dt}(x\circ \gamma_2)|_{t=0},
\end{equation}
for a coordinate chart $x$. It is checked easily that this definition is chart independent, i.e., the equivalent classes do not depend on the choice of the coordinate chart.
\begin{definition}
A tangent vector $v$ at a point $p\in \mathcal M$ is an equivalence class of curves on $\mathcal M$ that pass through $p$. The set of all tangent vectors at $p$ is denoted as $T_p\mathcal M$ and is called the tangent space of $\mathcal M$ at point $p$.
\end{definition}
It is well known that the $T_p\mathcal M$ forms an $m$-dimensional linear vector space \cite{robbin2011introduction}. It is shown in \cite{robbin2011introduction} that the vector space $T_p\mathcal M$ is isomorphic to the vector space of linear differential operators, directional derivatives, acting on the differentiable real-valued functions defined on an open neighborhood of $p$. Therefore, a basis for $T_p\mathcal M$ is often denoted as $\{ \partial /\partial x_i \}_{i=1}^{m}$ and a vector $v\in T_p\mathcal M$ can be written as 
\begin{equation} \label{eq:2.11}
	\sum_{i=1}^m v_i \frac{\partial}{\partial x_i},
\end{equation}
where $v_i\in \mathbb R$ for $i = 1,\dots,m$. Note that the symbol for partial derivative is rather symbolic since the direction $x_i$ is not defined explicitly on $T_p\mathcal M$. To compute a tangent vector for a given chart $x$ and a given smooth function $f:\mathcal M \to \mathbb R$ one reads
\begin{equation} \label{eq:2.12}
	\sum_{i=1}^m v_i \frac{\partial}{\partial x_i} (f) := \sum_{i=1}^m v_i \frac{\partial}{\partial x_i} (f \circ x^{-1}),
\end{equation}
where the partial derivatives appear on the right hand side are the conventional partial derivative operators on a Euclidean space.

It is often desirable to consider the dual space to $T_p\mathcal M$, denoted by $T_p^*\mathcal M$ and referred to as the \emph{cotangent space}. The natural isomorphism between $T_p\mathcal M$ and $T_p^*\mathcal M$ indicates that the cotangent space is also an $m$-dimensional linear vector space. Given $\{ \partial /\partial x_i \}$, a basis for $T_p\mathcal M$, a dual basis for $T_p^*\mathcal M$ is the set of basis vectors $\{ dx_i \}_{i=1}^{m}$ that satisfy the following
\begin{equation} \label{eq:2.13}
	dx_i(\frac{\partial}{\partial x_j}) = \delta_{i,j}, \quad i,j=1,\dots,m,
\end{equation}
where $\delta_{i,j}$ is the Kronecker's delta function.

To be able to define a vector field on a manifold, we need to assign a tangent vector to every point of a manifold. Such an object belongs to a structure that, informally, is obtained by glueing the tangent space $T_p\mathcal M$ to every point $p\in \mathcal M$. This structure is referred to as the \emph{tangent bundle}, denoted as $T \mathcal M$, and is defined as $T \mathcal M := \{ (p,v) | p\in \mathcal M, \ v \in T_p \mathcal M\}$.
\begin{theorem} \label{theorem:2.1}
\cite{robbin2011introduction} The tangent bundle $T\mathcal M$ is a smooth manifold.
\end{theorem}

\begin{proof}
We define the projection operator $\pi:T\mathcal M \to M$ as
\begin{equation} \label{eq:2.2}
	\pi : (p,v) \to p.
\end{equation}
It is checked easily that $\pi^{-1}(\{p\})$ is the $m$-dimensional linear vector space $T_{p} \mathcal M$. Now assume that $(x,U)$ is a coordinate chart for $\mathcal M$, such that $p\in U$. We construct a coordinate chart $(\bar x , \pi^{-1}(U) )$ for $T\mathcal M$ as follows 
\begin{equation} \label{eq:2.3}
	\begin{aligned}
		&\bar x : \pi^{-1}(U) \subset T \mathcal M \to \mathbb R^m \times \mathbb R^m, \\
		&\bar x : (p,v) \to (x(p),v_1,\dots,v_m),
	\end{aligned}
\end{equation}
where $v_1,\dots,v_m\in \mathbb R$ are components of $v$ in $\mathbb R^m$. It can be checked easily that for two intersecting coordinate charts $(x_\alpha,U_\alpha)$ and $(x_\beta, U_\beta)$, the transition map $\bar x_{\alpha} \circ (\bar x_\beta)^{-1}$ is a $C^{\infty}$ map. Thus, $T \mathcal M$ is a smooth manifold, and the mapping (\ref{eq:2.3}) suggests, it is $2m$-dimensional.
\end{proof}

In a similar fashion, we may obtain a smooth manifold by gluing the cotangent space $T^*_p \mathcal M$ to the manifold $\mathcal M$ to obtain the \emph{cotangent bundle}, denoted as $T^* \mathcal M$. 

\begin{definition}
continuous injective mappings $X:\mathcal M \to T \mathcal M$ and $X^*:\mathcal M \to T^* \mathcal M$ is called a \emph{vector field} and a \emph{co-vector field} of $\mathcal M$, respectively.
\end{definition}

\begin{definition}
Suppose that $X$ is a vector field on a smooth manifold $\mathcal M$. The smooth curve $c:(a,b)\to \mathcal M$ is called an \emph{integral curve} of $X$ passing through $p$ if $(a,b)$ is an open interval of $\mathbb R$ containing $0$ such that  $c(0)=p$ satisfying
\begin{equation} \label{eq:2.5}
	\frac{d}{dt}c(t) = X(c(t)), \quad \forall t\in(a,b) 
\end{equation}
\end{definition}
Given a coordinate chart $x$, one can solve \cref{eq:2.5} for $c$. It is known from the theory of ordinary differential equations that \cref{eq:2.5} has a unique solution \cite{teschl2012ordinary}. 
\begin{definition}
	The \emph{flow} of $X$ is a collection of maps $\varphi_t : \mathcal M \to \mathcal M$ such that the map $t \to \varphi_t(p)$ is an integral curve for some initial point $p\in \mathcal M$.
\end{definition}
Note that we have $\varphi_0 = id$, the identity map. Furthermore, that the uniqueness of integral curves implies the following important property of flows
\begin{equation} \label{eq:2.51}
	\varphi_{t+s} = \varphi_t \circ \varphi_s.
\end{equation}

In the study of MOR, transformation between vector spaces emerges naturally. In a more general setting, it is beneficial to study transformations between smooth manifolds. Later in this chapter, we discuss how some manifold structures can be preserved over such transformation which lays a foundation for geometric MOR.

Let $\mathcal M$ and $\mathcal N$ be an $m$ dimensional and an $n$ dimensional smooth manifolds, respectively. Furthermore, let $\phi:\mathcal M\to \mathcal N$ be a smooth mapping, i.e., if $(x,U)$ is a coordinate chart for $\mathcal M$ and $(y,V)$ is a coordinate chart for $\mathcal N$ such that $\phi(U)\cap V\neq \emptyset$, then the mapping $y\circ \phi \circ x^{-1}|_{\phi(U)\cap V}:\mathbb R^{m}\to \mathbb R^{n}$ is $C^{\infty}$. The \emph{differential map} of $\phi$ at a point $p\in \mathcal M$, denoted by $T_{p}\phi$, is a map between the tangent spaces $T_p \mathcal M$ and $T_{\phi(p)} \mathcal N$ defined as
\begin{equation} \label{eq:2.4}
	T_p \phi(v) = \frac{d}{dt}(\phi\circ \gamma(t))|_{t=0},
\end{equation}
for some tangent vector $v\in T_{p} \mathcal M$ and some curve $\gamma$ in the equivalence class of $v$. It can be shown that $T_p f$ only depends on $v$ and not the choice of the curve $\gamma$ \cite{robbin2011introduction}. The inverse function theorem \cite{rudin1976principles} indicates that if $\phi$ is a vector space isomorphism then there is a neighborhood $U$ of $p$ and a neighborhood $V$ of $\phi(p)$, such that $\phi:U\to V$ is a diffeomorphism.

\section{Tensors and Differential Forms} \label{section:2.2}
Often, quantities that appear in physics are linearly dependent on the vectors and convectors that describe them. Examples of such a quantity would be the measurement of the magnetic field with linearly independent directions of measurement \cite{Wald:106274}, and the strength of resistance in dissipative fluid flows. 

In the context of Hamiltonian systems and symplectic model reduction, we often require to align the flow of the system with respect to some potential. Differential forms and tensors, generalize the idea of inner-product and angle between vectors and convectors to the extent that  alignment of two vector field with respect to one another is possible.

\begin{definition}
A \emph{$k$-form} $\Omega$ on a manifold $\mathcal M$ is a function $\Omega(p): T_p\mathcal M \times \dots \times T_p\mathcal M \ (k \text{ times}) \to \mathbb R$ such that it is multilinear
\[
	\Omega(v_1,\dots,\alpha v_i + v'_i, \dots, v_k) = \alpha \Omega(v_1,\dots,v_i, \dots, v_k) + \Omega(v_1,\dots,v'_i, \dots, v_k),
\]
for $i=1,\dots,k$ and some $\alpha \in \mathbb R$, and it is skew-symmetric
\[
	\Omega(v_1,\dots,v_i,\dots,v_j, \dots, v_k) = - \Omega(v_1,\dots,v_j,\dots,v_i, \dots, v_k),
\]
for $i,j=1,\dots,k$ and $i\neq j$.
\end{definition}

\begin{definition}
A \emph{$(k,l)$-tensor} on a manifold $\mathcal M$ is a function 
\[
	\Lambda(p):T_p^*\mathcal M \times \dots \times T_p^*\mathcal M \ (k \text{ times}) \times T_p\mathcal M \times \dots \times T_p\mathcal M \ (l \text{ times}) \to \mathbb R
\]
That is multilinear.
\end{definition}
Therefore the Euclidean inner product is a $(0,2)$-tensor, a co-vector is a $(0,1)$-tensor, and any $k$-form is also a $(0,k)$-tensor. We denote by $\Lambda_{k}(\mathcal M)$ and $\Lambda_{k,l}(\mathcal M)$ the set of all differential $k$-forms and $(k,l)$-tensors, respectively.

Given a coordinate chart $x$ and a basis $\{ e_1,\dots,e_m\}$ for $T_{p}\mathcal M$, multilinearity of a $k$-form implies that
\begin{equation} \label{eq:2.6}
	\Omega(v_1,\dots,v_k) = \sum_{i_1,\dots,i_k =1}^{m} \omega_{i_1,\dots,i_k} v_{1}^{i_1}\dots v_{k}^{i_k}, 
\end{equation}
where $\omega_{i_1,\dots,i_k} = \Omega(e_{i_1},\dots,e_{i_k})$ and $v_{l}^{i_l}$ is the $i_l$th component of $v_l$ with respect to the coordinate chart $x$. Therefore, any $k$-form is completely described through $\Omega(e_{i_1},\dots,e_{i_k})$, for $i_1,\dots,i_k=1,\dots,m$. Simple calculation also shows a similar results for $(k,l)$-tensors \cite{Wald:106274}.

We now introduce some basic tensor operators, which allows us to construct higher order tensors and differential forms, from simpler building blocks.
\begin{definition}
	Let $\Gamma_1$ and $\Gamma_2$ be a $(k_1,l_l)$-tensor and a $k_2,l_2$-tensor, respectively. Their tensor product $\Gamma_1\otimes \Gamma_2$ is a $(k_1+k_2,l_1+l_2)$-tensor defined as
\[
	\begin{aligned}
	(\Gamma_1\otimes \Gamma_2)(v^*_1,\dots,&v^*_{k_1+k_2};w_1,\dots w_{l_1+l_2}) = \\
		&\Gamma_1(v^*_1,\dots,v^*_{k_1};w_1,\dots w_{l_1})\cdot \Gamma_1(v^*_{k_1+1},\dots,v^*_{k_1+k_2};w_{l_1+1},\dots w_{l_1+l_2}).
	\end{aligned}
\]
\end{definition}
To be able to construct differential forms from $(0,k)$-tensors, we need an operator that skew-symmetrizes tensors. The \emph{alternation operator}, is a tensor operator that achieves this and is defined as
\begin{equation} \label{eq:2.7}
	\textbf{A}(\Gamma)(v_1,\dots,v_l) = \frac{1}{p!} \sum_{\pi \in S_l} \text{sgn}(\pi) \Gamma(v_{\pi(1)},\dots,v_{\pi(l)}).
\end{equation}
Here $\Gamma$ is a $(0,l)$-tensor, $S_l$ is the permutation group of the set $\{1,\dots,l\}$ and sgn$(\pi)$ returns $1$ if $\pi$ is an even permutation, and $-1$ if $\pi$ is an odd permutation. It is easily checked that $\textbf{A}(\Gamma)$ is skew-symmetric. Therefore, $\textbf{A}$ constructs a differential form from a $(0,l)$-tensor.

Note that the tensor product of differential forms does not result a differential form, since the skew-symmetry will be lost. The \emph{wedge product} allows us to construct higher order differential forms while preserving the skew-symmetry. Let $\Omega_1$ and $\Omega_2$ be a $k_1$-form and a $k_2$ form, respectively. Then, their wedge product is a $(k_1+k_2)$-from defined as
\begin{equation} \label{eq:2.8}
	\Omega_1 \wedge \Omega_2 = \frac{k_1! + k_2!}{k_1!k_2!} \mathbf{A}(\Omega_1\otimes \Omega_2).
\end{equation}
It is well known that the wedge product is associative, bilinear and anti-commutative \cite{marsden2013introduction}. The following theorem states that any differential $k$-form can be written as a linear combination of wedge product of co-vectors. We refer the reader to \cite{} for the complete proof.
\begin{theorem} \label{theorem:2.2}
Any $k$-form $\Omega$ can be written locally as
\begin{equation} \label{eq:2.9}
	\Omega = \sum_{i_1<\dots<i_k} \omega_{i_1,\dots,i_k} dx_{i_1}\wedge\dots\wedge dx_{i_k}.
\end{equation}
\end{theorem}

%% ADD dw = 0 implies w_i = 0 for all i
It is easily verified that $\Omega = 0$ implies $\omega_{i_1,\dots,i_k} = 0$ for all $i_1<\dots<i_k$. Differential maps are often useful to transfer manifold structures from a known manifold to an unknown manifold. For example, $T_p\phi$ allows us to identify the tangent space of $\mathcal N$ at $\phi_p$ using the tangent space of $M$ at $p$. Similarly, we may use the differential maps to construct differential forms and tensors for unknown manifolds. 

\begin{definition}
Let $\phi:\mathcal M \to \mathcal N$ be a smooth manifold mapping and $\Omega$ be a differential $k$-form on $N$. Then the \emph{pull back} of $\Omega$ with $\phi$, is a $k$-form on $M$ denoted by $\phi^*\Omega$ defined as
\[
	(\phi^*\Omega)_p(v_1,\dots,v_k) = \Omega_{\phi(p)}( T_p\phi(v_1) , \dots , T_p\phi(v_k) ),
\]
for any $p\in \mathcal M$ and $v_1\dots,v_k \in T_p \mathcal M$. In case $\phi$ is a diffeomorphism, the \emph{push forward} operator is denoted as $\phi_*$ and is defined by $\phi_* = (\phi^{-1})^*$.
\end{definition}

A basic operator on tensors and differential forms, is the \emph{contraction} operator.
\begin{definition}
	Let $\Omega$ be a differential $k$-form and $X$ be a smooth vector field on a smooth manifold $\mathcal M$. The contraction of $\Omega$ with respect to $X$ is a $(k-1)$-form defined by
\[
	(\mathbf i_{X}\Omega)_p(v_1,\dots,v_{k-1}) = \Omega(X(p),v_1,\dots,v_k).
\]
\end{definition}
Note that the contraction operator is sometimes referred to as the \emph{interior product}.

There are multiple ways to generalize the notion of differentiation to the manifold setting. The challenge in defining a unique derivative operator that is consistent with the conventional derivative operator in calculus, is that a general manifold setting do not provide an algebraic method to compare points on a manifold. Differential forms, can be used to construct a relation, between tangent spaces that are in the locality of a point $p$ on a manifold. The following, is a generalization of the derivative notation to the manifold setting using differential forms.
\begin{definition}
Let $\mathcal M$ be a smooth manifold and $\Omega$ be a differential $k$-form on $\mathcal M$. Given a vector field $X$ on $\mathcal M$ with the flow map $\varphi_t$, the Lie derivative of $\Omega$ with respect to $X$ is defined as
\begin{equation} \label{eq:2.10}
	\mathcal L_X \Omega = \lim_{t\to 0} \frac 1 t (\varphi^*_t \Omega - \Omega) = \frac {d}{dt} \varphi^*_t |_{t=0}.
\end{equation}
\end{definition}
Note that the flow map can be viewed as a mapping $\varphi_t:\mathcal M \to \mathcal M$ and the differential map $T \varphi _t$ defines a mapping between $T_p \mathcal M$ and $T_{ \varphi_t(p) } \mathcal M$. Therefore, the Lie derivative measures infinitesimal differences in $\Omega$ when evaluated at $p$ and at $\varphi_t(p)$. The following theorem summarizes some basic properties of the Lie derivative.

\begin{theorem} \label{theorem:2.3}
Suppose that $X$ is a smooth vector field defined on a smooth manifold $\mathcal M$ with $\varphi$ the flow of $X$. Furthermore, suppose that $\Omega$ is a differential $k$-form. The following statements hold:
\begin{enumerate} [label=(\alph*)]
\item for a smooth scalar function $f:\mathcal M \to \mathbb R$, $\mathcal L_X f = X\cdot f$, where $X \cdot f$ is the directional derivative of $f$ along $X$.
\item (The Lie derivative formula)
\[
	\frac{d}{dt} \varphi^*_t \Omega = \varphi^*_t \mathcal L_X \Omega.
\]
\item in case $\Omega$ is a time dependent differential form, i.e. $\Omega = \Omega_t$, we have
\[
	\frac{d}{dt} \varphi^*_t \Omega_t = \varphi^*_t \mathcal L_X \Omega + \varphi^*_t \frac{d}{dt} \Omega_t.
\]
\end{enumerate}
\end{theorem}
\begin{proof}
(a) follows from the definition of the Lie derivative. To show (b), we have
\begin{equation} \label{eq:2.11}
	\frac{d}{dt} \varphi^*_t \Omega = \lim_{h\to 0} \frac 1 h( \varphi^*_{t+h} \Omega - \varphi^*_t \Omega ) = \varphi^*_t (\lim_{h\to 0} \frac 1 h (\varphi^*_{h} \Omega - \Omega) ) = \varphi^*_t \mathcal L_X \Omega,
\end{equation}
where the second equality is due to the fact that $\varphi_{t+h} = \varphi_t \circ \varphi_h$. To show (c), assume that vectors $v_1\dots,v_k\in T_{p}\mathcal M$ at some point $p\in \mathcal M$ are provided . We define the function $\alpha(t,s):\mathbb R^2 \to \mathbb R$ as $\alpha(t,s) = \varphi_t^*\Omega_s(v_1,\dots,v_k)$. Therefore, $\frac{d}{dt} \varphi^*_t \Omega_t|_p$ is the directional derivative of $\alpha(t,s)$ along the direction $(1,1)$. We have
\begin{equation} \label{eq:2.12}
\begin{aligned}
	D_{(1,1)} \alpha(t,s) = \partial_t \alpha(t,s) + \partial_s  \alpha(t,s) &= \partial_t \varphi^*_t \Omega_s|_{t=s} + \varphi^*_t (\partial_s \Omega_s)|_{t=s} \\
	&= \varphi^*_t \mathcal L_X \Omega_t + \varphi^*_t (\frac{d}{dt} \Omega_t).
\end{aligned}
\end{equation}
Here, $D_{(1,1)}$ is the conventional directional derivative operator and all terms are evaluated at the point $p$. In the second equality, we used statement (b) and also the linearity of $\varphi^*$. Since the choice of the point $p$ and vectors $v_1,\dots,v_k$ is arbitrary, then the formula holds independently. What is left, is to show that $\alpha(t,s)$ is a differentiable function. Given that $\Omega_t$ is continuously differentiable, we see from above that the partial derivatives of $\alpha(t,s)$ is continuous, thus, is differentiable. 
\end{proof}

For a $k$-form $\Omega$, the \emph{exterior derivative} is a $(k+1)$-form $\mathbf d \Omega$, that captures the differential changes in $\Omega$ and for a given coordinate chart $x$ is defined as
\begin{equation} \label{eq:2.13}
	\mathbf d \Omega =  \sum_j \sum_{i_1<\dots<i_k} \frac{\partial \alpha_{i_1,\dots,i_k}}{\partial x_j} dx_j\wedge dx_{i_1}\wedge\dots\wedge dx_{i_k}.
\end{equation}
Note that the above definition is shown to be to be chart independent \cite{rudin1964principles}. It is checked with a straightforward calculation that for $v_0,v_1\dots,v_k \in T_p\mathcal M$ and $p\in \mathcal M$ we have
\begin{equation} \label{eq:2.14}
	\mathbf d \Omega_p(v_0,v_1,\dots,v_k) = \sum_{i=1}^k (-1)^i \nabla \Omega \cdot v_i (v_0,\dots,\mathring{v_i},\dots,v_k), 
\end{equation}
where $\nabla \Omega \cdot v_i = \sum_j \sum_{i_1<\dots<i_k} v_i^j \partial \omega_{i_1,\dots,i_k}/\partial x_j$, $\mathring{v_i}$ means that the vector $v_i$ is dropped out, and $v_i^j$ is the $j$th component of $v_i$. A differential $k$-form $\Omega$ is called \emph{closed}, if $\mathbf d \Omega = 0$, and is called \emph{exact} there exist there is a differential $(k-1)$-form $\Gamma$ such that $\Omega = \mathbf d \Gamma$. The symmetry in partial derivatives implies that the exterior derivative of any differential form is closed, i.e., $\mathbf d^2 = 0$ for any differential $k$-form. However, no all differential forms are exact. We close this section by showing the sufficient condition for a differential form to be exact.

\begin{theorem} \label{theorem:2.4}
(Poincar\'e Lemma) Let $\mathcal M$ be a smooth $n$-dimensional manifold with $\Omega$ a differential $k$-form, defined on $\mathcal M$. If $\Omega$ is closed then for any point $p\in \mathcal M$, there is a neighborhood $U$ of $p$ where $\Omega|_{U}$ is exact.
\end{theorem}

\begin{proof}
Let $(x,U)$ be a coordinate cart around $p\in \mathcal M$. Without loss of generality, we assume that $x(p)=0$ and that the image of $U$ under $x$ contains an open ball around the origin. We show that the push forward of $\Omega$ to any point on this ball is exact.

The condition $\mathbf d \Omega = 0$ implies that
\begin{equation*}
	\sum_l \sum_{j_1<\dots<j_{k-1}} \frac{\partial \omega_{j_1,\dots,j_k}}{\partial x_l} dx_l\wedge dx_{j_1}\wedge \dots \wedge dx_{j_k} = 0, \quad j_1<\dots<j_k.
\end{equation*}
Note that the indices $l,j_1,\dots,j_k$ are not in the proper order. It is easily checked a reordering of the indices in the above expression yields
\begin{equation*}
	\sum_{n_1<\dots<n_{k+1}}\lambda_{n_1,\dots,n_{k+1}} dx_{n_1}\wedge \dots \wedge dx_{n_{k+1}} = 0,
\end{equation*}
with
\begin{equation*}
	\lambda_{n_1,\dots,n_{k+1}} = \sum_{m=1}^{k+1} (-1)^{m+1} \frac{\partial \omega_{n_1\dots\mathring n_m \dots n_{k+1}}}{\partial x_{n_m}},
\end{equation*}
which implies $\lambda_{n_1,\dots,n_{k+1}} = 0$, for all $n_1<\dots<n_{k+1}$. Here $\mathring n_m$ means that $n_m$ is omitted. Now we construct a $(k-1)$-form $\Gamma$ such that $\mathbf d \Gamma = \Omega$:
\begin{equation*}
	\Gamma_{(x_1,\dots,x_n)} = \left( \int_0^1 t^{k-1} \omega_{ji_1\dots,i_{k-1}}(tx_1,\dots,tx_n) x_j \ dt \right) dx_{i_1}\wedge \dots \wedge dx_{i_{k-1}}.
\end{equation*}
Taking the exterior derivative of $\Gamma$ gives
\begin{equation*}
	\mathbf d \Gamma = \left( \int_0^1 t^{k-1} \frac{\partial }{\partial x_l} \big( \omega_{ji_1\dots,i_{k-1}}(tx) x_j \big) \ dt \right) dx_l\wedge dx_{i_1}\wedge \dots \wedge dx_{i_{k-1}}.
\end{equation*}
Here $tx$ denotes the point $(tx_1,\dots,tx_n)$. We denote by $c_{li_1\dotsi_{k-1}}$ the coefficients in this expression. Similar to above, we can construct a reordering of the indices to obtain the coefficient of $dx_{j_1}\wedge\dots\wedge dx_{j_k}$ for $j_1<\dots<j_{k}$ as
\begin{equation*}
	\sum_{m=1}^{k} (-1)^{m+1} c_{j_mj_1\dots\mathring j_{m}\dots j_k}.
\end{equation*}
It follows
\begin{equation*}
\begin{aligned}
\sum_{m=1}^{k} &(-1)^{m+1} c_{j_mj_1\dots\mathring j_{m}\dots j_k} \\
	& = -\sum_{m=1}^k (-1)^m \int_0^1 t^{k-1} \frac{\partial}{\partial x_{j_m}}\big( \omega_{jj_1\dots\mathring j_m\dots j_k}(tx) x_j \big)\ dt \\
	& = -\sum_{m=1}^k (-1)^m \int_0^1 t^{k-1} \big( tx_j\frac{\partial \omega_{jj_1\dots\mathring j_m\dots j_k} }{\partial x_{j_m}}(tx) + \omega_{j_mj_1\dots\mathring j_m\dots j_k}(tx)  \big)\ dt \\
       & = \int_0^1 t^{k-1} \left( x_j t \sum_{m=1}^{k} (-1)^{m+1}\frac{\partial \omega_{jj_1\dots\mathring j_m\dots j_k}}{\partial x_{j_m}}(tx) + k \omega_{j_1\dots j_k}(tx)  \right) \ dt \\
       & = \int_0^1 t^{k-1} \left( x_j k t \frac{\partial \omega_{j_1\dots j_k}}{\partial x_j}(tx) + kt^{k-1} \omega_{j_1\dots j_k}(tx)\right)\ dt \\
       & = \int_0^1 \frac{\partial}{\partial t}(t^k \omega_{j_1\dots j_k}(tx) )\ dt \\
      	& = \omega_{j_1\dots j_k}(x).
\end{aligned}
\end{equation*}
Here, we used the condition $\lambda_{n_1,\dots,n_{k+1}} = 0$ and the chain rule to obtain the result.
\end{proof}


%\begin{proof}
%We construct an operator $\alpha:\Lambda_k\to \Lambda_{k-1}$ such that  $\mathbf d \circ \alpha + \alpha \circ \mathbf d = id$. This yields the result since, $\mathbf d^2 \Omega = 0$ and $\mathbf d( \alpha (\Omega) ) = \Omega$. Let (x,U) be a coordinate chart in the neighborhood of $p\in U$ such that $x(p)=(x_1,\dots,x_n)$ we construct $\alpha$ as follows.
%\begin{equation}
%	\alpha(\Omega_p)(v_1,\dots,v_{k-1}) := \int_0^1 t^{k-1} \Omega_{tx(p)}(v_1,\dots,v_{k-1}) \ dt,
%\end{equation}
%for some tangent vectors $v_1,\dots,v_{k-1}$.We first compute $\mathbf d \alpha(\Omega_p)$.
%\begin{equation}
%\begin{aligned}
%	\mathbf d \alpha(\Omega_p) (v_1,\dots,v_k) &= \sum_{i=1}^k (-1)^{i+1} \nabla \alpha(\Omega_p) \cdot v_i (v_0,\dots,\mathring{v_i},\dots,v_k) \\
%	&= \sum_{i=1}^k (-1)^{i+1} \int_{0}^1 t^{k-1} \Omega_{tx(p)} (v_0,\dots,\mathring{v_i},\dots,v_k) \ dt \\
%	& + \sum_{i=1}^k (-1)^{i+1} \int_0 ^1 t^k \nabla \Omega_{tx(p)} \cdot v_i (v_0,\dots,\mathring{v_i},\dots,v_k) \ dt.
%\end{aligned}
%\end{equation}
%Here, we used the product rule. Furthermore, note that given that $\Omega$ is smooth and bounded for $t\in[0,1]$, it is allowed to interchange the derivative and the integral operator. For $\alpha(\mathbf d (\Omega))$ we have
%\begin{equation}
%\begin{aligned}
%	\alpha(\mathbf d\Omega_p)(v_1,\dots,v_{k-1}) &= \int_0^1 t^{k-1} \mathbf d \Omega_{tx(p)}(v_1,\dots,v_{k-1}) \ dt \\
%	& = \int_0^1 t^{k-1} \sum_{i=1}^{k} (-1)^{i}\nabla \Omega_{tx(p)} \cdot v_i (v_1,\dots,\mathring{v}_i,v_{k-1}) \ dt
%\end{aligned}
%\end{equation}
%
%\end{proof}

 

\section{Hamiltonian Systems on a Symplectic Manifold} \label{section:2.3}
It is often useful to describe small changes in a state of a system with respect to some potential, or a vector field. Hamiltonian systems are systems where the changes in the state of the system is under the influence of a \emph{Hamiltonian vector field}. Such systems appear frequently in quantum physics, particle physics, celestial mechanics, cosmology, fluid mechanics, and classical mechanics. Conserved quantities, e.g. the system energy, are at the core of the dynamics of these systems. Consequently, the integral curve of these systems is aligned with the Hamiltonian vector field such that these quantities are conserved.

Differential forms, are tools that allow us to align an integral curve with one or more vector fields. To study Hamiltonian systems, we need to study basic features of differential 2-forms.

\begin{definition}
	Let $\mathcal M$ be a smooth manifold and $p\in \mathcal M$. The differential 2-form $\Omega_p$ is called \emph{non-degenerate} if $\Omega_p(v_1,v_2)=0$, for all $v_2\in T_p \mathcal M$, implies that $v_1 = 0$. $\Omega$ is called non-degenerate, if $\Omega_p$ is non-degenerate for all $p\in \mathcal M$.
\end{definition}
If a non-zero vector $v\in T_p \mathcal M$ is given, then $\Omega_p^{\flat}(v):=\Omega_p(v,\cdot):T_p \mathcal M \to \mathbb R$ can be viewed as a co-vector. Therefore, a non-degenerate $\Omega_p$ constructs an injective map $\Omega_p^{\flat}:T_p\mathcal M \to T_p^* \mathcal M$. When $\Omega_p^{\flat}$ is also surjective then it is said to be \emph{strongly non-degenerate}. 

\begin{definition}
	Let $\mathcal P$ be a smooth $m$-dimensional manifold and $\Omega$ be a closed, non-degenerate 2-form defined on $\mathcal P$. The pair $(\mathcal P,\Omega)$ is called a \emph{symplectic manifold}.
\end{definition}
Note that the condition for $\Omega$ to be closed is indeed required to construct a well-defined symplectic manifold. In the following theorem, we will see that a closedness of the 2-form is required to locally approximate a symplectic manifold with a symplectic linear vector space. This is especially important in the context of symplectic MOR where we approximate a high dimensional symplectic manifold with a low dimensional linear vector space.

\begin{theorem} \label{theorem:2.5}
(Darboux' Theorem) Let $\mathcal M$ be a manifold and $\Omega_1$ and $\Omega_2$ be two strongly non-degenerate and closed 2-forms defined on $\mathcal M$, such that $\Omega_1 = \Omega^2$, exactly at some $p\in \mathcal M$. Then there are neighborhoods $U$ and $V$ of $p$ such that the mapping $\varphi:U\to V$ is a diffeomorphism with $\varphi^* \Omega_2 = \Omega_1$
\end{theorem}

\begin{proof}
The idea is to construct a family of continuously varying 2-forms:
\begin{equation*}
\Omega_t = (1-t)\Omega_1 + t\Omega_2 = \Omega_1 + t \Omega,
\end{equation*}
where $\Omega = \Omega_2 - \Omega_1$. Now we would like to find a smooth vector field $X_t$ with the flow $\varphi_t$ such that $\frac{d}{dt} \varphi_t^*\Omega_t = 0$. We construct $U$ so small that $\Omega_t$ is strongly non-degenerate. This can be done since $\Omega_0 = \Omega_1$ and is constant at $p$, so the compactness of $[0,1]$ implies that there is an open ball around $p$ such that $\Omega_t$ is strongly non-degenerate for all $t\in [0,1]$. We have
\begin{equation*}
\begin{aligned}
	\frac{d}{dt} \varphi_t^*\Omega_t &= \varphi_t^* \mathcal L_{X_t} \Omega_t + \varphi_t^* \frac{d}{dt} \Omega_t \\
	&= \varphi_t^*( \mathbf d i_{X_t}\Omega_t + i_{X_t} \mathbf d \Omega_t ) + \varphi_t^* \Omega \\
	&= \varphi_t^*( \mathbf d i_{X_t}\Omega_t + \Omega ).
\end{aligned}
\end{equation*}
Here we used the Lie derivative formula for time dependent differential forms, Cartan's magic formula, closedness of $\Omega_t$, and the linearity of pull back operator. Since $\Omega$ is a closed form and strongly non-degenerate, we can apply the Poincar\'e lemma \ref{?} to write $\Omega = \mathbf d \Gamma$ for some 1-form $\Gamma$. Now the above expression is equal to
\begin{equation*}
	= \varphi_t^*( \mathbf d ( i_{X_t}\Omega_t + \Gamma ) ).
\end{equation*}
Therefore, it is sufficient to define $X_t$ to be the vector field associate to the relation $i_{X_t}\Omega_t = - \Gamma$. Note that the non-degeneracy of $\Omega_t$ in $U$ guarantees the uniqueness of $X_t$. Thus, $\varphi_1^* \Omega_1 = \varphi_0^* \Omega_0 = \Omega_0$
\end{proof}

\begin{corollary} \label{theorem:2.6}
Let $(\mathcal P, \Omega)$ b a symplectic manifold. There is a local coordinate chart $(x,U)$ around each point $z\in\mathcal P$ in which $\Omega$ is constant.
\end{corollary}
\begin{proof}
Take $\Omega_1 = \Omega$ and $\Omega_2 = \Omega_z$, i.e. $\Omega_2$ is a constant differential form. The rest follows from the proof of the Darboux' theorem. The constructed flow $\varphi_t$ provides a coordinate cart that transforms $\Omega$ into the constant form $\Omega_2$.
\end{proof}

\begin{corollary} \label{theorem:2.7}
Let $(\mathcal P, \Omega)$ be a finite-dimensional symplectic manifold. Then $\mathcal P$ is even dimensional and we can find a local coordinate chard $(x,U)$ with $x=(e_1,\dots,e_n,f_1,\dots,f_n)$ around each point $z\in \mathcal P$ such that
\begin{equation} \label{eq:2.15}
	\Omega = de_i\wedge df_i
\end{equation}
\end{corollary}
\begin{proof}
(The symplectic Gram-Schmidt) Suppose that $(x,U)$ is a local coordinate chart around $z$ provided by the Darboux' theorem such that $\Omega_z$ is constant, i.e., $\Omega_z = \Omega$. Let $e_1\in T_z\mathcal P$ be a nonzero tangent vector. Non-degeneracy of $\Omega$ implies that there is a vector $f_1\in T_z\mathcal P$ such that $\Omega(e_1,f_1) = c_1 \neq 0$. We can swap and scale $e_1$ and $f_1$ to guarantee that $c_1 = 1$. Let $E_1 = \text{span}\{e_1,f_1\}$ and $E_2 = \{v\in T_z\mathcal P | \Omega(v,e) = 0,~ \forall e \in E_1\}$. It is easily verified that $E_1\cap E_2 = \emptyset$. Furthermore, for any $v\in T_z\mathcal P$, $v - \bar v \in E_2$ where $\bar v = -\Omega(v,f_1)e_1 + \Omega(v,e_1)f_1 \in E_1$, therefore $T_z\mathcal P = E_1 \oplus E_2$. Finally, since $\Omega$ is non-degenerate, it is also non-degenerate on $E_2$ as a subspace of $T_z\mathcal P$. Thus, we can continue inductively. Since $\mathcal P$ is finite dimensional this process ends, e.g., after $n$ steps. Furthermore, the sequence of basis vectors $A = \{ e_1,\dots,e_n,f_1,\dots,f_n\}$ forms a basis for $T_z\mathcal P$. In this basis, $\Omega$ takes the \emph{canonical} form
\begin{equation} \label{eq:2.141}
	\Omega( a, b) = a^T\mathbb J_{2n}b, \quad \mathbb J_{2n} =
	\begin{pmatrix}
		0_n & I_n \\
		-I_n & 0_n
	\end{pmatrix}.
\end{equation}
Here, $a,b\in \mathbb R^{2n}$ with $a_i = dx_i(v)$ and $b_i = dx_i(u)$, $i=1,\dots,2n$ for any $u,v \in T_z\mathcal P$ which results the form in \cref{eq:2.15}.
\end{proof}

Note that due to non-degeneracy of $\Omega$, vectors $f_1,\dots,f_n$ can be interpreted as co-vectors such that $\Omega^{\flat}(e_i)=f_i$, for $i=1,\dots,n$.  

\begin{corollary} \label{theorem:2.8}
Let $(\mathcal P, \Omega)$ be a symplectic manifold. There is a neighborhood $U$ around each point $z\in \mathcal P$ where $U$ is diffeomophic to a linear vector space $V$ where $\Omega$ is constant on $V$.
\end{corollary}
\begin{proof}
The proof is a direct consequence of the Darboux' theorem and \Cref{theorem:2.7}.
\end{proof}

\begin{definition}
Let $(\mathcal P_1,\Omega_1)$ and $(\mathcal P_2,\Omega_2)$ be two symplectic manifolds. The transformation $\varphi:\mathcal P_1 \to \mathcal P_2$ is called to be a \emph{symplectic transformation} if
\begin{equation}
	\varphi^* \Omega_2 = \Omega_1.
\end{equation}
\end{definition}

Now we are ready to define Hamiltonian systems on a symplectic Manifold.

\begin{definition}
Let $(\mathcal P,\Omega)$ be a symplectic manifold. We refer to a vector field $X_H$ a Hamiltonian vector field, if we can find a real function $H:\mathcal P \to \mathbb R$ such that
\begin{equation}
	i_{X_H}\Omega = \mathbf d H.
\end{equation}
We call $H$ the \emph{Hamiltonian function}. In this case, the equations of evolution is given by
\begin{equation}
	\dot z = X_H(z),
\end{equation}
and is referred to as \emph{Hamilton's equation of evolution}.
\end{definition}
When $\mathcal P$ is $2n$-dimensional and a canonical basis is provided, Hamilton's equation take the form
\begin{equation}
	\frac{dq_i}{dt} = \frac{\partial H}{\partial p_i}, \quad \frac{dp_i}{dt} = - \frac{\partial H}{\partial q_i}, \quad i=1,\dots,n .
\end{equation}

The following proposition shows that the flow of a Hamiltonian vector field is a symplectic map.
\begin{proposition} \label{theorem:2.9}
Let $\varphi_t$ be the flow of a Hamiltonian vector field $X_H$. Then $\varphi_t:\mathcal P \to \mathcal P$ is a symplectic transformation.
\end{proposition}
\begin{proof}
We have
\begin{equation*}
	\frac{d}{dt} \varphi_t^* \Omega = \varphi_t^* \mathcal L_{X_H} \Omega = \varphi_t^*( i_{X_H} \mathbf d \Omega + \mathbf d i_{X_H} \Omega ) =  \varphi_t^*( i_{X_H} \mathbf d \Omega + \mathbf d^2 H ).
\end{equation*}
However, closedness of $\Omega$ implies $\mathbf d \Omega = 0$. Furthermore, since any exact differential form is closed, then $\mathbf d^2 \Omega = 0$. Thus, $\frac{d}{dt} \varphi_t^* \Omega = 0$ and $\varphi_t^* \Omega = \varphi_0^* \Omega = \Omega$.
\end{proof}
\begin{corollary} \label{theorem:2.10}
The Hamiltonian is conserved along the Hamiltonian flow.
\end{corollary}
\begin{proof}
I follows that
\begin{equation*}
\begin{aligned}
	\frac{d}{dt} H(\varphi_t(z)) = \mathbf dH\left( \frac{d}{dt} \varphi_t(z) \right) = \mathbf dH \left( X_H(z) \right) &= i_{X_H(z)} \Omega(X_H(z))\\
	& = \Omega(X_H(z),X_H(z)) = 0.
\end{aligned}
\end{equation*}
Therefore, $H\circ \varphi_t$ is constant in time.
\end{proof}

\section{Hamiltonian systems on a symplectic linear vector space} \label{section:2.4}
In the symplectic model reduction, the flow of a Hamiltonian system is projected on a low dimensional symplectic linear vector space. Therefore, a deeper study of symplectic linear vector spaces is essential in the understanding of symplectic MOR. In this section we assume that a symplectic manifold $(\mathcal P, \Omega)$ is also a linear vector space, i.e., we identify both $\mathcal P$ and $T_z \mathcal P$ with a linear vector space $\mathcal Z$.

\begin{definition}
Let $\mathcal Z$ be a finite dimensional linear vector space with $\Omega$ a constant 2-form defined on $Z$. The pair $(\mathcal Z,\Omega)$ is called a symplectic linear vector space if $\Omega$ is non-degenerate.
\end{definition}
Note that \Cref{theorem:2.7} indicates that that $\mathcal Z$ is even-dimensional. Furthermore, the symplectic Gram-Schmidt process can construct a canonical basis, in which $\Omega(a,b) = a^T \mathbb J_{2n} b$, where $\mathbb J_{2n}$ is defined in (\ref{eq:2.141}) and $a,b\in \mathcal Z$. However, in a non-canonical basis, the symplectic form $\Omega$ takes the form $\Omega(a,b) = a^TJ_{2n}b$, where $J_{2n}$ is a full rank and skew-symmetric matrix. Therefore, when the matrix form of $\Omega$ is being discussed, we may refer to $\Omega$ as $\mathbb J_{2n}$ or $J_{2n}$, respectively.

A symplectic linear vector space can be equipped with an inner product to form a inner product space. However, a symplectic basis is not necessarily well-conditioned (or orthonormal) with respect to a general inner-product. The following proposition provides a natural inner product defined on a symplectic linear vector space that carries the conditioning of a canonical basis.

\begin{proposition} \label{theorem:2.11}
Let $(\mathcal Z,\Omega)$ be a $2n$-dimensional symplectic linear vector space. Furthermore, let $A = \{ e_1,\dots,e_n,f_1,\dots,f_n \}$ be a canonical basis for $\mathcal Z$ with respect to $\Omega$. Then there exists an inner product operator $<,>:\mathcal Z\times \mathcal Z \to \mathbb R$, such that $A$ is orthonormal.
\end{proposition}

\begin{proof}
Consider the linear mapping $\Omega^{\flat}:\mathcal Z \to \mathcal Z^*$ and let $e_i^*$ and $f_i^*$ be the image of $e_i$ and $f_i$ under $\Omega^{\flat}$, respectively for $i=1,\dots,n$. Non-degeneracy of $\Omega^{\flat}$ indicates that $e_i^*$ and $f_i^*$ are unique and non-zero, and that $A^*=\{e_1^*,\dots,e_n^*,f_1^*,\dots,f_n^*\}$ forms a basis for $\mathcal Z^*$. We construct a linear map $T:\mathcal Z^* \to \mathcal Z$ by prescribing its action on the basis co-vectors of $A^*$ as
\begin{equation*}
	T(e_i^*) = -f_i, \quad T(f_i^*) = e_i, \quad i=1,\dots,n.
\end{equation*}
Note that $T$ is a bijective. Now we define the operator $<,>:\mathcal Z\times \mathcal Z \to \mathbb R$ as
\begin{equation*}
	<u,v> = T^{-1}_{u} (v),	
\end{equation*}
where $T^{-1}_{u} = T^{-1}(u) \in \mathcal Z^*$. It is straight forward to check that $<,>$ is symmetric, bilinear and positive-definite. Therefore, it defines an inner-product on $\mathcal Z$. Furthermore, we have
\begin{equation*}
	<e_i,e_j> = \delta_{ij},\quad <f_i,f_j> = \delta_{ij}, \quad <e_i,f_j> = 0, \quad i,j=1,\dots,n.
\end{equation*}
\end{proof}

\begin{corollary} \label{theorem:2.12}
When $\mathbb J_{2n}$ is the matrix form of $\Omega$, then $<,>$ construction in \Cref{theorem:2.11} is the Euclidean inner product. The canonical basis constructed with respect to $\mathbb J_{2n}$, in this case, is called an \emph{ortho-symplectic} basis.
\end{corollary}

An inner product defined in \Cref{theorem:2.11} is especially useful since one can switch between computing the 2-form $\Omega$ or the inner production $<,>$, when needed. However, when dealing with a subspace of $\mathcal Z$, further structures are required to guarantee that such inner-product exists.

\begin{definition} \cite{abraham1978foundations}
Let $(\mathcal Z, \Omega)$ be a symplectic linear vector space and $\mathcal E \subset \mathcal Z$ be a subspace. The \emph{symplectic complement} of $\mathcal E$, referred to as $\mathcal E^{\perp}$, is a linear subspace defined as
\begin{equation*}
	\mathcal E^{\perp} = \{ a \in \mathcal Z | \Omega(a,b) = 0, \ \text{for all } b \in \mathcal E \}.
\end{equation*}
Furthermore, we call $\mathcal E$ an \emph{isotropic subspace} if $\mathcal E\subset \mathcal E^{\perp}$, and a \emph{symplectic subspace} if $\Omega|_{\mathcal E\times \mathcal E}$ is non-degenerate. Furthermore, we call $\mathcal E$ a \emph{Lagrangian subspace}, if both $\mathcal E$ and $\mathcal E^C$ are isotropic, where $\mathcal E^C$ is vector space complement of $\mathcal E$\footnote{$E^{C}$ is a vector space complement of $E$ if $\mathcal{Z} = \mathcal E\oplus \mathcal E^C$. }.
\end{definition}
\begin{proposition} \label{theorem:2.13} \cite{abraham1978foundations}
Let $(\mathcal Z, \Omega)$ be a symplectic linear vector space and $\mathcal E\subset \mathcal F$ is a linear subspace. Then the following are equivalent:
\begin{enumerate} [label=(\alph*)]
\item $\mathcal E$ is a Lagrangian subspace.
\item $\mathcal E = \mathcal E^{\perp}$.
\item $\mathcal E$ is isotropic and dim$(\mathcal E) = \frac 1 2 \text{dim}(\mathcal Z)$.
\end{enumerate}
\end{proposition}
\begin{proof}
We first prove that (a) implies (b). Since a Lagrangian subspace is isotropic then $\mathcal E \subset \mathcal E^{\perp}$. Let $F = E^C$ and take Take $v \in \mathcal E^{\perp}$. It is easily verified that $\mathcal Z = \mathcal E \oplus\mathcal F$. Therefore, $v = e + f$, where $e \in \mathcal E$ and $f\in \mathcal F$. Since $\mathcal E$ is isotropic then $\Omega(e,u) = 0$ for all $u \in \mathcal E$. Since $v\in \mathcal E^{\perp}$, it follows for any $u\in E$ that
\begin{equation*}
	0 = \Omega(v,u) = \Omega(e,u) + \Omega(f,u) = \Omega(f,u).
\end{equation*}
But since $\mathcal F$ is isotropic then $\Omega(f,u) = 0$ for all $u \in \mathcal F$. This implies that $\Omega(f,u) = 0$ for all $u\in\mathcal E \cup \mathcal F = \mathcal Z$. Since $\Omega$ is non-degenerate, it implies that $f=0$ and $v=e\in \mathcal E$. Therefore $\mathcal E^{\perp} \subset \mathcal E$.

Now we show that (b) implies (c). Consider the mapping $\Omega^{\flat}:\mathcal Z \to \mathcal Z^*$. Note that $\mathcal E^{\perp}$ is in the kernel of $\Omega^{\flat}|_{\mathcal E} : \mathcal E \to (\mathcal Z \backslash \mathcal E^{\perp})^*$. Since $\Omega^{\flat}$ is injective, then dim$(\mathcal E) \leq \text{dim}(\mathcal Z) - \text{dim}(\mathcal E^{\perp})$. Furthermore, $\Omega^{\flat}|_{\mathcal E}$ can be viewed as a mapping from $\mathcal Z$ to $\mathcal E^{*}$ where the kernel is exactly $\mathcal E^{\perp}$. Therefore, dim$(\mathcal E) \geq \text{dim}(\text{range}(\Omega^{\flat}|_{\mathcal E})) = \text{dim}(\mathcal{Z}) - \text{dim}(\mathcal E^{\perp})$. The two inequalities implies dim$(\mathcal Z) = \text{dim}(\mathcal E) + \text{dim}(\mathcal E^{\perp})$ which together with (b) provides the result.

Finally, we show that (c) implies (a). Note that from what we derived above we can conclude that dim$(\mathcal E) = \text{dim}(\mathcal E^{\perp})$. But since $\mathcal E \subset \mathcal E^{\perp}$, then $\mathcal E = \mathcal E^{\perp}$. Now we show that $\mathcal F = \mathcal E^{C}$ is isotropic. Take $f_1 \notin \mathcal E$, and define $\mathcal F_1 = \text{span}\{f_1\}$. Since $\mathcal E \cap \mathcal F_1 = \emptyset$, it follows\footnote{For subsets $E$ and $F$ we have $(E\cap F)^{\perp} = (E^{\perp \perp} \cap F^{\perp \perp})^{\perp} =( F^\perp + E^{\perp})^{\perp \perp} = F^{\perp} + E^\perp$ \cite{abraham1978foundations}.} that $\mathcal E + \mathcal F_1^{\perp} = \mathcal Z$. Therefore, we can pick $f_2\in \mathcal F_1^{\perp}$ such that $f_2\notin \mathcal E$, and define $\mathcal F_2 = \text{span}\{f_2\} + \mathcal F_1$. We continue this process inductively to construct $\mathcal F_n$ such that $\mathcal Z = \mathcal E \oplus \mathcal F_n$. It follows that 
\begin{equation*}
\begin{aligned}
	\mathcal F_n = \text{span}\{f_1,\dots,f_n\} & \subset \text{span}(f_1)^{\perp} \cap \dots \cap \text{span}(f_n)^{\perp} \\
	&= (\text{span}(f_1) + \dots + \text{span}(f_n))^{\perp} =  \mathcal F_n^{\perp}.
\end{aligned}
\end{equation*}
The second step is due to the fact that each $f_i$ is taken from $\mathcal F_{i-1}^{\perp}$. Therefore $\mathcal F = \mathcal F_n$ is isotropic and this completes the proof.
\end{proof}

\Cref{theorem:2.13} suggests that a Lagrangian subspace is a maximal isotropic subspace. We are now ready to state the following theorem.

\begin{theorem} \label{theorem:2.14}
Let $(\mathcal Z, \Omega)$ be a $2n$-dimensional symplectic linear vector space. Furthermore suppose that $\mathcal X\subset \mathcal Z$ is $2k$-dimensional a linear subspace with $k<n$. If the pair $(\mathcal X,\Omega|_{\mathcal X})$ contain a Lagrangian subspace, then $(\mathcal X,\Omega|_{\mathcal X})$ is also a symplectic linear vector space.
\end{theorem}
\begin{proof}
We show that $\Omega|_{\mathcal X}$ is non-degenerate on $\mathcal X$. Let $\mathcal E$ be a Lagrangian subspace of $\mathcal X$. Since $\mathcal F = \mathcal E^{C}$ and dim$(F) = \text{dim}(E) = \text{dim}(\mathcal X)/2$ then by \Cref{theorem:2.13} (c), $\mathcal F$ is also a Lagrangian subspace. Furthermore, \Cref{theorem:2.13} implies that $\mathcal E = \mathcal E^{\perp}$ and $\mathcal F = \mathcal F^{\perp}$. Now assume that there is $u\in\mathcal X$ such that $\Omega|_{\mathcal X}(u,v)=0$ for all $v\in \mathcal X$. Then $v\in \mathcal E^{\perp} = \mathcal E$ and $v\in \mathcal F^{\perp} = \mathcal F$. Therefore $v\in\mathcal E \cap \mathcal F = \{ 0 \}$.
\end{proof}

Once a Lagrangian subspace is identified in linear subspace of a symplectic linear vector space, one can induce a reduced symplectic form $\Omega|_{\mathcal X}$. This is particularly important in the context of model reduction where a low dimensional symplectic reduced vector space is constructed. The induced symplectic form $\Omega|_{\mathcal X}$ can then be used to construct an ortho-symplectic basis on the low dimensional symplectic subspace.
