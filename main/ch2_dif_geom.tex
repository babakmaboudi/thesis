\chapter{Manifolds and Differential Forms}

In preparation for later chapters, we recall basic facts regarding differential geometry and symplectic geometry. Although the applications of MOR presented in this thesis are mostly described on a linear vector space, the theory of smooth manifold can provide a great insight into the geometric MOR. The main aim of this chapter is to introduce the concepts of ``symmetry'' and ``structure'' which play an important role in the Lagrangian and Hamiltonian mechanics. Exploiting these symmetries and structures in MOR can provide robustness and long-time stability in the reduced system.

\section{Smooth Manifolds}
Let $\mathcal M$ to be a topological Hausdorff \cite{friedman1970foundations} set. In the neighborhood $U$ of $p\in \mathcal M$, if there is a bijective map $x:U\to V\subset \mathbb R^{m}$, for some positive integer $m$, then $(x,U)$ is called a \emph{coordinate chart}. $\mathcal M$ is called a \emph{smooth manifold} if there exists a set of coordinate charts $\{(x_{\alpha},U_{\alpha})\}_{\alpha\in I}$ such that $\{U_{\alpha}\}_{\alpha\in I}$ covers $\mathcal M$ and that the mapping $x_{\alpha_2}\circ (x_{\alpha_1})^{-1}:\mathbb R^{m} \to \mathbb R^{m}$ is a $C^{\infty}$ mapping, for any $\alpha_1,\alpha_2\in I$. The integer $m$ is called the \emph{dimension} of $\mathcal M$ if $x_{\alpha}(U_{\alpha}) \subset \mathbb R^{m}$ for all $\alpha \in I$. Throughout this thesis, we assume that there exists a coordinate chart $(x,U)$ that is global, i.e., $U$ covers $\mathcal M$.

Tangent vectors to a manifold allow us to describe the local dynamics of a particle moving on a smooth manifold. There are multiple ways to define tangent vectors. The most intuitive way however, uses curves defined on a smooth manifold.

A $C^{\infty}$ mapping $\gamma:\mathbb R \to \mathcal M$ is called a curve on $\mathcal M$ passing through $p\in \mathcal M$ if $\gamma(t) = p$, for some $t\in \mathbb R$. Without loss of generality, we may assume that $\gamma$ passes through $p$ at $t=0$. We define an equivalence relation between smooth curves that pass through $p$ as follows: $\gamma_1$ and $\gamma_2$, two smooth curves on $\mathcal M$, are equivalent if
\begin{equation} \label{eq:2.1}
	\frac{d}{dt}(x\circ \gamma_1)|_{t=0} = \frac{d}{dt}(x\circ \gamma_2)|_{t=0},
\end{equation}
for a coordinate chart $x$. It is checked easily that this definition is chart independent, i.e., the equivalent classes do not depend on the choice of the coordinate chart.
\begin{definition}
A tangent vector $v$ at a point $p\in \mathcal M$ is an equivalence class of curves on $\mathcal M$ that pass through $p$. The set of all tangent vectors at $p$ is denoted as $T_p\mathcal M$ and is called the tangent space of $\mathcal M$ at point $p$.
\end{definition}
It is well known that the $T_p\mathcal M$ forms an $m$-dimensional linear vector space \cite{robbin2011introduction}. It is shown in \cite{robbin2011introduction} that the vector space $T_p\mathcal M$ is isomorphic to the vector space of linear differential operators, directional derivatives, acting on the differentiable real-valued functions defined on an open neighborhood of $p$. Therefore, a basis for $T_p\mathcal M$ is often denoted as $\{ \partial /\partial x_i \}_{i=1}^{m}$ and a vector $v\in T_p\mathcal M$ can be written as 
\begin{equation}
	\sum_{i=1}^m v_i \frac{\partial}{\partial x_i},
\end{equation}
where $v_i\in \mathbb R$ for $i = 1,\dots,m$. Note that the symbol for partial derivative is rather symbolic since the direction $x_i$ is not defined explicitly on $T_p\mathcal M$. To compute a tangent vector for a given chart $x$ and a given smooth function $f:\mathcal M \to \mathbb R$ one reads
\begin{equation}
	\sum_{i=1}^m v_i \frac{\partial}{\partial x_i} (f) := \sum_{i=1}^m v_i \frac{\partial}{\partial x_i} (f \circ x^{-1}),
\end{equation}
where the partial derivatives appear on the right hand side are the conventional partial derivative operators on a Euclidean space.

It is often desirable to consider the dual space to $T_p\mathcal M$, denoted by $T_p^*\mathcal M$ and referred to as the \emph{cotangent space}. The natural isomorphism between $T_p\mathcal M$ and $T_p^*\mathcal M$ indicates that the cotangent space is also an $m$-dimensional linear vector space. Given $\{ \partial /\partial x_i \}$, a basis for $T_p\mathcal M$, a dual basis for $T_p^*\mathcal M$ is the set of basis vectors $\{ dx_i \}_{i=1}^{m}$ that satisfy the following
\begin{equation}
	dx_i(\frac{\partial}{\partial x_j}) = \delta_{i,j}, \quad i,j=1,\dots,m,
\end{equation}
where $\delta_{i,j}$ is the Kronecker's delta function.

To be able to define a vector field on a manifold, we need to assign a tangent vector to every point of a manifold. Such an object belongs to a structure that, informally, is obtained by glueing the tangent space $T_p\mathcal M$ to every point $p\in \mathcal M$. This structure is referred to as the \emph{tangent bundle}, denoted as $T \mathcal M$, and is defined as $T \mathcal M := \{ (p,v) | p\in \mathcal M, \ v \in T_p \mathcal M\}$.
\begin{theorem}
\cite{robbin2011introduction} The tangent bundle $T\mathcal M$ is a smooth manifold.
\end{theorem}

\begin{proof}
We define the projection operator $\pi:T\mathcal M \to M$ as
\begin{equation} \label{eq:2.2}
	\pi : (p,v) \to p.
\end{equation}
It is checked easily that $\pi^{-1}(\{p\})$ is the $m$-dimensional linear vector space $T_{p} \mathcal M$. Now assume that $(x,U)$ is a coordinate chart for $\mathcal M$, such that $p\in U$. We construct a coordinate chart $(\bar x , \pi^{-1}(U) )$ for $T\mathcal M$ as follows 
\begin{equation} \label{eq:2.3}
	\begin{aligned}
		&\bar x : \pi^{-1}(U) \subset T \mathcal M \to \mathbb R^m \times \mathbb R^m, \\
		&\bar x : (p,v) \to (x(p),v_1,\dots,v_m),
	\end{aligned}
\end{equation}
where $v_1,\dots,v_m\in \mathbb R$ are components of $v$ in $\mathbb R^m$. It can be checked easily that for two intersecting coordinate charts $(x_\alpha,U_\alpha)$ and $(x_\beta, U_\beta)$, the transition map $\bar x_{\alpha} \circ (\bar x_\beta)^{-1}$ is a $C^{\infty}$ map. Thus, $T \mathcal M$ is a smooth manifold, and the mapping (\ref{eq:2.3}) suggests, it is $2m$-dimensional.
\end{proof}

In similar fashion, we may obtain a smooth manifold by gluing the cotangent space $T^*_p \mathcal M$ to the manifold $\mathcal M$ to obtain the \emph{cotangent bundle}, denoted as $T^* \mathcal M$. 

\begin{definition}
continuous injective mappings $X:\mathcal M \to T \mathcal M$ and $X^*:\mathcal M \to T^* \mathcal M$ is called a \emph{vector field} and a \emph{covector field} of $\mathcal M$, respectively.
\end{definition}

\begin{definition}
Suppose that $X$ is a vector field on a smooth manifold $\mathcal M$. The smooth curve $c:(a,b)\to \mathcal M$ is called an \emph{integral curve} of $X$ passing through $p$ if $(a,b)$ is an open interval of $\mathbb R$ containing $0$ such that  $c(0)=p$ satisfying
\begin{equation} \label{eq:2.5}
	\frac{d}{dt}c(t) = X(c(t)), \quad \forall t\in(a,b) 
\end{equation}
\end{definition}
Given a coordinate chart $x$, one can solve \cref{eq:2.5} for $c$. It is known from the theory of ordinary differential equations that \cref{eq:2.5} has a unique solution \cite{teschl2012ordinary}. 
\begin{definition}
	The \emph{flow} of $X$ is a collection of maps $\varphi_t : \mathcal M \to \mathcal M$ such that the map $t \to \varphi_t(p)$ is an integral curve for some initial point $p\in \mathcal M$.
\end{definition}
Note that from the uniqueness of integral curves we deduce an important property of flows
\begin{equation}
	\varphi_{t+s} = \varphi_t \circ \varphi_s.
\end{equation}

\section{Tensors and Differential Forms}
Often, quantities that appear in physics are linearly dependent on the vectors and convectors that describe them. Examples of such a quantity would be the measurement of the magnetic field with linearly independent directions of measurement \cite{Wald:106274}, and the strength of resistance in dissipative fluid flows. 

In the context of Hamiltonian systems and symplectic model reduction, we often require to align the flow of the system with respect to some potential. Differential forms and tensors, generalize the idea of inner-product and angle between vectors and convectors to the extent that  alignment of two vector field with respect to one another is possible.

\begin{definition}
A \emph{$k$-form} $\Omega$ on a manifold $\mathcal M$ is a function $\Omega(p): T_p\mathcal M \times \dots \times T_p\mathcal M \ (k \text{ times}) \to \mathbb R$ such that it is multilinear
\[
	\Omega(v_1,\dots,\alpha v_i + v'_i, \dots, v_k) = \alpha \Omega(v_1,\dots,v_i, \dots, v_k) + \Omega(v_1,\dots,v'_i, \dots, v_k),
\]
for $i=1,\dots,k$ and some $\alpha \in \mathbb R$, and it is skew-symmetric
\[
	\Omega(v_1,\dots,v_i,\dots,v_j, \dots, v_k) = - \Omega(v_1,\dots,v_j,\dots,v_i, \dots, v_k),
\]
for $i,j=1,\dots,k$ and $i\neq j$.
\end{definition}

\begin{definition}
A \emph{$(k,l)$-tensor} on a manifold $\mathcal M$ is a function 
\[
	\Lambda(p):T_p^*\mathcal M \times \dots \times T_p^*\mathcal M \ (k \text{ times}) \times T_p\mathcal M \times \dots \times T_p\mathcal M \ (l \text{ times}) \to \mathbb R
\]
That is multilinear.
\end{definition}
Therefore the Euclidean inner product is a $(0,2)$-tensor, a covector is a $(0,1)$-tensor and a $1$-form, and any $k$-form is also a $(0,k)$-tensor.

Given a coordinate chart $x$ and a basis $\{ e_1,\dots,e_m\}$ for $T_{p}\mathcal M$, multilinearity of a $k$-form implies that
\begin{equation}
	\Omega(v_1,\dots,v_k) = \sum_{i_1,\dots,i_k =1}^{m} \omega_{i_1,\dots,i_k} v_{1}^{i_1}\dots v_{k}^{i_k}, 
\end{equation}
where $\omega_{i_1,\dots,i_k} = \Omega(e_{i_1},\dots,e_{i_k})$ and $v_{l}^{i_l}$ is the $i_l$th component of $v_l$ with respect to the coordinate chart $x$. Therefore, any $k$-form is completely described through $\Omega(e_{i_1},\dots,e_{i_k})$, for $i_1,\dots,i_k=1,\dots,m$. Simple calculation also shows a similar results for $(k,l)$-tensors \cite{Wald:106274}.

We now introduce some basic tensor operators, which allows us to construct higher order tensors and differential forms, from simpler building blocks.
\begin{definition}
	Let $\Gamma_1$ and $\Gamma_2$ be a $(k_1,l_l)$-tensor and a $k_2,l_2$-tensor, respectively. Their tensor product $\Gamma_1\otimes \Gamma_2$ is a $(k_1+k_2,l_1+l_2)$-tensor defined as
\[
	\begin{aligned}
	(\Gamma_1\otimes \Gamma_2)(v^*_1,\dots,&v^*_{k_1+k_2};w_1,\dots w_{l_1+l_2}) = \\
		&\Gamma_1(v^*_1,\dots,v^*_{k_1};w_1,\dots w_{l_1})\cdot \Gamma_1(v^*_{k_1+1},\dots,v^*_{k_1+k_2};w_{l_1+1},\dots w_{l_1+l_2}).
	\end{aligned}
\]
\end{definition}
To be able to construct differential forms from $(0,k)$-tensors, we need an operator that skew-symmetrizes tensors. The \emph{alternation operator}, is a tensor operator that achieves this and is defined as
\begin{equation}
	\textbf{A}(\Gamma)(v_1,\dots,v_l) = \frac{1}{p!} \sum_{\pi \in S_l} \text{sgn}(\pi) \Gamma(v_{\pi(1)},\dots,v_{\pi(l)}).
\end{equation}
Here $\Gamma$ is a $(0,l)$-tensor, $S_l$ is the permutation group of the set $\{1,\dots,l\}$ and sgn$(\pi)$ returns $1$ if $\pi$ is an even permutation, and $-1$ if $\pi$ is an odd permutation. It is easily checked that $\textbf{A}(\Gamma)$ is skew-symmetric. Therefore, $\textbf{A}$ constructs a differential form from a $(0,l)$-tensor.

Note that the tensor product of differential forms does not result a differential form, since the skew-symmetry will be lost. The \emph{wedge product} allows us to construct higher order differential forms while preserving the skew-symmetry. Let $\Omega_1$ and $\Omega_2$ be a $k_1$-form and a $k_2$ form, respectively. Then, their wedge product is a $(k_1+k_2)$-from defined as
\begin{equation}
	\Omega_1 \wedge \Omega_2 = \frac{k_1! + k_2!}{k_1!k_2!} \mathbf{A}(\Omega_1\otimes \Omega_2).
\end{equation}
It is well known that the wedge product is associative, bilinear and anti-commutative \cite{marsden2013introduction}. The following theorem states that any differential $k$-form can be written as a linear combination of wedge product of covectors. We refer the reader to \cite{} for the complete proof.
\begin{theorem}
Any $k$-form $\Omega$ can be written locally as
\begin{equation}
	\Omega = \sum_{i_1<\dots<i_k} \alpha_{i_1,\dots,i_k} dx_{i_1}\wedge\dots\wedge dx_{i_k}.
\end{equation}
\end{theorem}

For a $k$-form $\Omega$, the \emph{exterior derivative} is a $(k+1)$-form $\mathbf d \Omega$, that captures the differential changes in $\Omega$ and for a given coordinate chart $x$ is defined as
\begin{equation}
	\mathbf d \Omega =  \sum_j \sum_{i_1<\dots<i_k} \frac{\partial \alpha_{i_1,\dots,i_k}}{\partial x_j} dx_{i_1}\wedge\dots\wedge dx_{i_k}.
\end{equation}
Note that the above definition is shown to be to be chart independent and that $\mathbf d (\mathbf d \Omega) = 0$ for any $k$-form $\Omega$. We refer the reader to \cite{} for a detailed proof.

Another basic operator on tensors and differential forms, is the \emph{contraction} operator.
\begin{definition}
	Let $\Omega$ be a differential $k$-form and $X$ be a smooth vector field on a smooth manifold $\mathcal M$. The contraction of $\Omega$ with respect to $X$ is a $(k-1)$-form defined by
\[
	(\mathbf i_{X}\Omega)_p(v_1,\dots,v_{k-1}) = \Omega(X(p),v_1,\dots,v_k).
\]
\end{definition}
Note that the contraction operator is sometimes referred to as the \emph{interior product}.

In the study of MOR, transformation between vector spaces emerges naturally. In a more general setting, it is beneficial to study transformations between smooth manifolds. Later in this chapter, we discuss how some manifold structures can be preserved over such transformation which lays a foundation for geometric MOR.

Let $\mathcal M$ and $\mathcal N$ be an $m$ dimensional and an $n$ dimensional smooth manifolds, respectively. Furthermore, let $\phi:\mathcal M\to \mathcal N$ be a smooth mapping, i.e., if $(x,U)$ is a coordinate chart for $\mathcal M$ and $(y,V)$ is a coordinate chart for $\mathcal N$ such that $\phi(U)\cap V\neq \emptyset$, then the mapping $y\circ \phi \circ x^{-1}|_{\phi(U)\cap V}:\mathbb R^{m}\to \mathbb R^{n}$ is $C^{\infty}$. The \emph{differential map} of $\phi$ at a point $p\in \mathcal M$, denoted by $T_{p}\phi$, is a map between the tangent spaces $T_p \mathcal M$ and $T_{\phi(p)} \mathcal N$ defined as
\begin{equation} \label{eq:2.4}
	T_p \phi(v) = \frac{d}{dt}(\phi\circ \gamma(t))|_{t=0},
\end{equation}
for some tangent vector $v\in T_{p} \mathcal M$ and some curve $\gamma$ in the equivalence class of $v$. It can be shown that $T_p f$ only depends of $v$ and not the choice of the curve $\gamma$ \cite{robbin2011introduction}. The inverse function theorem \cite{rudin1976principles} indicates that if $\phi$ is a vector space isomorphism then there is a neighborhood $U$ of $p$ and a neighborhood $V$ of $\phi(p)$, such that $\phi:U\to V$ is a diffeomorphism.

Differential maps are often useful to transfer manifold structures from a known manifold to an unknown manifold. For example, $T_p\phi$ allows us to identify the tangent space of $\mathcal N$ at $\phi_p$ using the tangent space of $M$ at $p$. Similarly, we may use the differential maps to construct differential forms and tensors for unknown manifolds. This can be done by the means of \emph{pull back} and \emph{push forward} of a differential form using the transformation $\phi$.

\begin{definition}
Let $\phi:\mathcal M \to \mathcal N$ be a smooth manifold mapping and $\Omega$ be a differential $k$-form on $N$. Then the pull back of $\Omega$ with $\phi$, is a $k$-form on $M$ denoted by $\phi^*\Omega$ defined as
\[
	(\phi^*\Omega)_p(v_1,\dots,v_k) = \Omega( T_p\phi(v_1) , \dots , T_p\phi(v_k) ),
\]
for any $p\in \mathcal M$ and $v_1\dots,v_k \in T_p \mathcal M$. In case $\phi$ is a diffeomorphism, the push forward operator is denoted as $\phi_*$ and is defined by $\phi_* = (\phi^{-1})^*$.
\end{definition}

\section{Hamiltonian Systems on a Symplectic Manifold}
It is often useful to describe small changes in a state of a system with respect to some potential, or a vector field. Hamiltonian systems are systems where the changes in the state of the system is under the influence of a \emph{Hamiltonian vector field}. Such systems appear frequently in quantum and particle physics, celestial mechanics and cosmology, fluid mechanics, and classical mechanics. Conserved quantities, e.g. the system energy, are at the core of the dynamics of these systems. Consequently, the integral curve of these systems is aligned with the Hamiltonian vector field such that these quantities are conserved.

Differential forms, are tools that allow us to align an integral with one or more vector fields. To study Hamiltonian systems, we need to study basic features of differential 2-forms.

\begin{definition}
	Let $\mathcal M$ be a smooth manifold and $p\in \mathcal M$. The differential 2-form $\Omega_p$ is called \emph{non-degenerate} if $\Omega_p(v_1,v_2)=0$, for all $v_2\in T_p \mathcal M$, implies that $v_2 = 0$. $\Omega$ is called non-degenerate, $\Omega_p$ is non-degenerate for all $p\in \mathcal M$.
\end{definition}
If a non-zero vector $v\in T_p \mathcal M$ is given, then $\Omega_p^{\flat}(v):=\Omega_p(v,\cdot):T_p \mathcal M \to \mathbb R$ can be viewed as a covector. Therefore, a non-degenerate $\Omega_p$ constructs an injective map $\Omega_p^{\flat}:T_p\mathcal M \to T_p^* \mathcal M$. When $\Omega_p^{\flat}$ is also surjective then it is said to be \emph{strongly non-degenerate}.

\begin{definition}
	Let $\mathcal P$ be a smooth manifold and $\Omega$ be a closed, non-degenerate 2-form defined on $P$. The pair $(\mathcal P,\Omega)$ is called a \emph{symplectic manifold}.
\end{definition}


