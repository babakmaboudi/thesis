\chapter{Model Order Reduction} \label{chapter:3}
Mathematical simulation is now evermore important in engineering, science, and related domains, thanks to expeditious advances in computational sciences and rapid growth in computational capacities over the past decades. Numerical evaluation of partial differential equations lies at the core of these disciplines which accommodates design, optimization, and prediction of inputs and outputs of interest. However, the exceeding need for accuracy, and the complexity of multi-physical new applications makes conventional approaches for solving large scale systems of partial differential equations impractical. 

To cope with these limitations, \emph{reduced-order} modeling (ROM), appose to \emph{full-order} or \emph{high-fidelity} modelling, has been an area of active research for the past decade. These methods eliminate the redundant physical or computation dimensions of the full-order differential equation to construct a low dimensional reduced-order system. This approximation in return significantly accelerates the evaluation of the system. Reduced basis (RB) methods, are among the most successful ROMs and is vastly used in academia and industry. RB methods, seek a low dimensional reduced subspace that accurately represents the full-order solution. Confining the system on this subspace, using a projection, can then accelerate the evaluation of the system. These methods are particularly successful in reducing the computational complexity of \emph{parametric PDEs} or multi-query systems, where a system of differential equations needs to be evaluated for a large number of input parameters.

In this chapter we summarize the fundamentals of MOR and especially RB methods. We present various conventional approaches and algorithms for linear and nonlinear problems. Since time, as a parameter, is particularly important in the context of Hamiltonian systems, we will develop this chapter with an emphasis on time-dependent problems.

\section{new}
