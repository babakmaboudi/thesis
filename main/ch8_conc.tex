\chapter{Conclusions} \label{chapter:8}

During the past decades, the need to solve complex, multi-physics, and multi-scale applications have become central in science, engineering and across many industrial domains. The numerical evaluation of such models using classical approaches, however, is often prohibitive due to limitations in computational capacities. In such situations, model order reduction is playing an increasingly important role in advancements in scientific computing and high-performance computing by reducing the intrinsic computational complexity of many modern models.

Despite the success of model order reduction for elliptic and parabolic PDEs, model reduction for hyperbolic systems remains a challenge. Symmetries, invariants, and conservation laws are a fundamental feature of such models, and are often destroyed during model order reduction. The violation of such features not only result in an inaccurate model, but may also cause numerical instabilities in the reduced order model.

This thesis studies and develops model order reduction techniques that conserve certain invariants and symmetries of hyperbolic systems of PDEs. Conserving such structures not only result in a physically meaningful reduced model, but provides robust long time behaviour and a stable reduced model.

To achieve this goal, we study model order reduction from a geometric point of view. The crucial role of time is highlighted for the construction of symmetry-preserving model order reduction. We furthermore investigate why conventional model order reduction techniques often break the symmetries of hyperbolic problems.

Hamiltonian systems, as a special case of highly symmetric PDEs, are intensively studied in this thesis. We discuss how the symplectic structure, the symmetry of Hamiltonian systems, can be conserved in model order reduction. A greedy approach for construction of a reduced basis is presented. And we discuss how a symplectic Galerkin projection constructs a reduced Hamiltonian system that carries the symmetries of the original Hamiltonian system. The reduced Hamiltonian, as an approximation to the original Hamiltonian, is a conserved quantity for the reduced system. Hence, the loss in the Hamiltonian due to model order reduction remains constant and can be controlled.

To adapt the symplectic model reduction to an unstructured numerical discretization, the method is coupled with a weighted norm. A reduced system is constructed by orthogonally projecting a generalized Hamiltonian system onto the reduced space, with respect to a weighted inner product. The reduced system, however, carries the Hamiltonian structure and also the symplectic symmetry. It is shown that the new method can be viewed as a natural extension of the symplectic model reduction, and therefore retains the structure preserving features, e.g. symplecticity and stability.

In many applications in engineering, models appear as a dissipative perturbation of Hamiltonian system. In such models, the Hamiltonian systems is no longer symplectic. In this thesis, we consider a canonical extension of dissipative Hamiltonian systems, by coupling the dissipative system with a canonical heat bath, resulting a closed and conservative system. A symplectic model reduction method can then be applied to conserve the symmetries of the extended model, and, consequently, conserve the evolution of energy and dissipation at the level of the reduced system. It is shown that the extension of the system does not pose a significant additional computational burden.

The numerical experiments in this thesis illustrate that the proposed methods consistently result in a robust reduced system with excellent stability. Conventional model reduction techniques, even when the reduced basis is chosen to yield a high accuracy, may yield an unstable or poorly performing reduced system. Numerical experiments confirm that the conservation of symmetries can significantly enhance the overall dynamics of the reduced system.

To generalize the symplectic model reduction to more complex problems, a conservative model reduction technique of fluid flow is proposed. Skew-symmetric models for fluid flow are well-known for conserving quadratic invariants of a fluid flow in a numerical evaluation. The key ingredient in these methods is the construction of a discrete skew-symmetric operator. A proper model order reduction method preserves the skew-symmetry of such differential operators. This helps to define quadratic invariants in the reduced system that approximate the quadratic invariants of the high fidelity system. Numerical experiments suggest that the skew-symmetric form consistently yields a robust reduced system over long time-integration, even when the reduced model does not represent the high-fidelity solution accurately. 

What is less emphasised in this thesis is the question of reducibility of general hyperbolic problems. Transport of information, potentially throughout the entire domain, is a distinctive feature of hyperbolic problems. This often covers patterns in the ensemble of snapshots of the system and inhibits the possibility of describing the system as a linear combination of a relatively few basis vectors. The construction of efficient reduced order models for such cases, therefore, can be a possible extension to this work.

Although conservation of symplectic symmetry and quadratic invariants is discussed intensively in this thesis, conservation of general conservation laws or invariants is to be investigated. An extension of symplectic model order reduction may be to seek the conservation of the Poisson structure, or the multi-symplectic structure, on a symplectic manifold. In addition, the conservation of integral curves over model order reduction, in order to recover a stable reduced system, remains future work.

This thesis provides a promising approach to the construction of robust, accurate, physically meaning-full reduced system for Hamiltonian systems and fluid flow. It is also extends the understanding of what can be achieved with model order reduction and reduced basis methods. Indeed, the conservation of general nonlinear invariants, e.g. for Hamiltonian systems, in a linearly transformed and approximated system is a key achievement. 

Modeling is the art of approximately describing nature with understandable tools. Therefore, constructing a simplified and reduced model that resembles the symmetries and distinctive features and invariants of a complex system is another step in mathematical modelling. What is presented in this thesis highlights the potential role of structure-preserving model reduction in the future advancements of modeling.
