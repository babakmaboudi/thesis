%\begingroup
%\let\cleardoublepage\clearpage


% English abstract
\cleardoublepage
\chapter*{Abstract}
%\markboth{Abstract}{Abstract}
\addcontentsline{toc}{chapter}{Abstract (English/Français/Deutsch)} % adds an entry to the table of contents

During the past decade, model order reduction (MOR) has been successfully applied to reduce the computational complexity of elliptic and parabolic systems of partial differential equations (PDEs). However, MOR of hyperbolic equations remains a challenge. Symmetries and conservation laws, which are a distinctive feature of such systems, are often destroyed by conventional MOR techniques, resulting in a perturbed and often unstable reduced system. The goal of this thesis is to study and develop model order reduction techniques that can preserve nonlinear invariants, symmetries, and conservation laws and to understand the stability properties of these methods compared to conventional techniques. Hamiltonian systems, as systems that are driven by symmetries, are studied intensively from the point of view of MOR. Furthermore, a conservative model reduction of fluid flow is presented. It is illustrated that conserving invariants, conservation laws, and symmetries not only result in a physically meaningful reduced system, but also result in an accurate and robust reduced system with enhanced stability.

% German abstract
%\begin{otherlanguage}{german}
%\cleardoublepage
%\chapter*{Zusammenfassung}
%%\markboth{Zusammenfassung}{Zusammenfassung}
% put your text here
%\lipsum[1-2]
%\end{otherlanguage}




% French abstract
\begin{otherlanguage}{french}
\cleardoublepage
\chapter*{Résumé}
Au cours de la derni\`ere d\'ecennie, la r\'eduction d'ordre de mod\`ele (ROM) a r\'eussi \`a r\'eduire la complexit\'e de calcul des syst\`emes elliptiques et paraboliques d'\'equations aux d\'eriv\'ees partielles (EDP). Cependant, ROM des \'equations hyperboliques reste un d\'efi. Les sym\'etries et les lois de conservation, qui sont une caract\'eristique distinctive de tels syst\`emes, sont souvent d\'etruites par les techniques conventionnelles de ROM qui aboutissent \`a un syst\`eme r\'eduit perturb\'e et souvent instable. Le but de cette th\`ese est d'\'etudier et de d\'evelopper des techniques de r\'eduction d'ordre de mod\`ele pouvant pr\'eserver les invariants nonlin\'eaires, les sym\'etries et les lois de conservation et de comprendre les propri\'et\'es de stabilit\'e de ces m\'ethodes par rapport aux techniques conventionnelles. Les syst\`emes Hamiltoniens, en tant que syst\`emes pilot\'es par des sym\'etries, sont \'etudi\'es de mani\'ere intensive depuis le point de vue ROM. De plus, une r\'eduction mod\'er\'ee du d\'ebit de fluide est pr\'esent\'ee. Il est illustr\'e que la conservation des invariants, des lois de conservation et des sym\'etries non seulement aboutit \`a un syst\`eme r\'eduit physiquement significatif, mais construit \'egalement un syst\`eme r\'eduit robuste avec une stabilit\'e accrue.
\end{otherlanguage}


%\endgroup			
%\vfill
